%%%
\documentclass[3p]{elsarticle}
\usepackage{graphicx}
\usepackage{amssymb}
\usepackage{subfiles}
\usepackage{subcaption}
\usepackage{amsmath}
%\usepackage{hyperref}
%\usepackage{lineno}
%\usepackage[T1]{fontenc}

\usepackage{floatrow}
\usepackage[colorlinks]{hyperref}

%%%%%%%%%
%\makeatletter
%\let\c@author\relax
%\makeatother
%\let\bibhang\relax
%\let\citename\relax
%\let\bibfont\relax
%\let\Citeauthor\relax
%\expandafter\let\csname ver@natbib.sty\endcsname\relax
%
%\usepackage[
%	style=phys,
%	backend=bibtex8,
%	isbn=false,
%    	url=false,
%%	natbib=true,
%    	doi=false]{biblatex}
%%\bibliographystyle{elsarticle-num}
%\bibliography{CLReview_bib.bib}
%
%\renewbibmacro{in:}{}
%
%%%%%%%%%


\newcommand{\etal}{{\it et al.}}
\newcommand{\beq}{\begin{equation}}
\newcommand{\eeq}{\end{equation}}
\newcommand{\bea}{\begin{eqnarray}}
\newcommand{\eea}{\end{eqnarray}}
\newcommand{\ev}[1]{\left\langle #1 \right\rangle}
\renewcommand{\d}{{\rm d}}

% Way to reference equations consistently - if we don't want to use it this is fine.
\newcommand{\equref}[1]{Eq.~(\ref{Eq:#1})}
\newcommand{\figref}[1]{Fig.~\ref{fig:#1}}
\newcommand{\secref}[1]{Sec.~\ref{sect:#1}}

\def\CP{{\mathcal P}}
\def\CC{{\mathcal C}}
\def\CW{{\mathcal W}}
\def\CO{{\mathcal O}}
\def\CZ{{\mathcal Z}}
\def\CD{{\mathcal D}}
\def\del{{\nabla}}

%%%%%%%%%%%%%%%

\newcommand*{\doi}[1]{\href{http://dx.doi.org/#1}{doi: #1}}

% Draft stuff, take this out for real version.
\usepackage[dvipsnames]{xcolor}
\newcommand{\lukas}[1]{{\color{Cerulean} #1}}
\newcommand{\draftnote}[1]{{\color{red} \bf[NOTE: #1]}}
\newcommand{\casey}[1]{{\color{Green} #1}}
\newcommand{\jed}[1]{{\color{Blue} #1}}
\usepackage{soul}
\newcommand{\cancel}[1]{\st{#1}}

%%%%%%%%%%%%%%%%%%%%%%%%%%%%%%%%%%%%%%%%%%%%
\journal{\ }


\begin{document}

\begin{frontmatter}

\title{Complex Langevin and other approaches to \\ the sign problem in quantum many-body physics}

\author[unc]{C. E.~Berger}
\ead{cberger3@live.unc.edu}

\author[tud,gsi]{L.~Rammelm\"uller}
\ead{lrammelmueller@theorie.ikp.physik.tu-darmstadt.de}

\author[unc]{A. C.~Loheac}
\ead{loheac@live.unc.edu}

\author[tud]{F.~Ehmann}
\ead{fehmann@theorie.ikp.physik.tu-darmstadt.de}

\author[tud,emmi,fair]{J.~Braun}
\ead{jens.braun@physik.tu-darmstadt.de}

\author[unc]{J. E.~Drut}
\ead{drut@email.unc.edu}

\address[unc]{Department of Physics and Astronomy, University of North Carolina, Chapel Hill, North Carolina 27599, USA}
\address[tud]{Institut f\"ur Kernphysik (Theoriezentrum), Technische Universit\"at Darmstadt, D-64289 Darmstadt, Germany}
\address[gsi]{GSI Helmholtzzentrum f\"ur Schwerionenforschung GmbH, Planckstra\ss e 1, D-64291 Darmstadt, Germany}
\address[fair]{FAIR, Facility for Antiproton and Ion Research in Europe GmbH, Planckstra\ss e 1, D-64291   Darmstadt, Germany}
\address[emmi]{ExtreMe Matter Institute EMMI, GSI, Planckstra{\ss}e 1, D-64291 Darmstadt, Germany}

%%%%%%%%%%%%%%%%%%%%%
\begin{abstract}
We review the theory and applications of complex stochastic quantization to the quantum many-body problem.
Along the way, we present a brief overview of a number of ideas that either ameliorate or in some cases altogether solve the sign
problem, including the classic reweighting method, alternative Hubbard-Stratonovich transformations, dual variables (for
bosons and fermions), Majorana fermions, density-of-states methods, imaginary asymmetry approaches, and
Lefschetz thimbles. We discuss some aspects of 
the mathematical underpinnings of conventional stochastic quantization, provide a few pedagogical examples,
and summarize open challenges and practical solutions for the complex case. Finally, we review the recent applications of complex Langevin
to quantum field theory in relativistic and nonrelativistic quantum matter, with an emphasis on the nonrelativistic case.
\end{abstract}
%%%%%%%%%%%%%%%%%%%%%

\begin{keyword}
Sign problem \sep stochastic quantization \sep complex Langevin %algorithm
\end{keyword}

\end{frontmatter}

\tableofcontents

%\linenumbers
\newpage
\subfile{./1introduction/introduction.tex}
%\newpage
\subfile{./2generalformalism/generalformalism.tex}
%\newpage
\subfile{./3mathunderpinnings/mathunderpinnings.tex}
%\newpage
\subfile{./4applications-REL/relativistic.tex}
%\newpage
\subfile{./5applications-NREL/nonrelativistic.tex}
%\newpage
\subfile{./6outlook/outlook.tex}

%%%%%%%%%%%%%%%%%%%%%%%%
\section*{Acknowledgements}

We are especially grateful to G. Aarts, F. Attanasio, G. Basar, S. Chandrasekharan, P. de Forcrand, C. Gattringer, E. Huffman, K. Langfeld, and E. Seiler
for their extremely useful comments on earlier versions of this manuscript.
We would also like to acknowledge G. Aarts, C. Gattringer, C. Ratti, and H. Singh for kindly allowing us to reproduce their figures.

This material is based upon work supported by the National Science Foundation under
Grant No. PHY{1452635} (Computational Physics Program).
C.E.B. acknowledges support from the United States Department of Energy through the
Computational Science Graduate Fellowship (DOE CSGF) under grant number DE-FG02-97ER25308.
J.B. acknowledges support by the DFG under grant BR 4005/4-1 (Heisenberg program). J.B. and F.E. acknowledge 
support by the DFG grant BR 4005/5-1. 
J.B. and L.R. acknowledge support by HIC for FAIR within the LOEWE program of the State of Hesse.



%%%%%%%%%%%%%%%%%%%%%%%%
\section*{References}
%\bibliographystyle{elsarticle-num}
\bibliographystyle{mybst}
\bibliography{CLReview_bib.bib}
%\printbibliography

\end{document}
