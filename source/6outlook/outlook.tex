\documentclass[../main.tex]{subfiles}
\begin{document}
\section{\label{sect:outlook}Summary and Outlook}

In this report we have discussed the origin and methods to address the sign problem as it appears across a wide range of
systems in relativistic and nonrelativistic quantum many-body physics. We have proposed a broad classification of the various methods into
{\it new-variables}, {\it statistical}, and {\it complex-plane} approaches. After a brief overview of each of those categories,
we focused on the complex form of stochastic quantization, namely CL.

Complex Langevin first appeared in the 80s as a natural generalization of stochastic quantization for cases with a complex action.
Operationally, stochastic quantization does not require a probability to be defined (there is no Metropolis accept/reject step), which made field
complexification (itself needed in the presence of a sign or complex phase problem) look more like a feature than a bug. From the
mathematical standpoint, however, the challenges seemed daunting. It took decades for the community to begin to understand the properties and
behavior of CL, to clarify the origin of its problems and limitations, and to propose solutions. Some of that progress was enabled by advances in
hardware, as modern personal computers are powerful enough to run small but useful quantum field theory calculations with little wait; %(while on a train or sitting at a coffee shop)
that was certainly not the case in the 80s, nor in the 90s when initial explorations of this method were underway.

As computer power continues to grow, and given the overall progress made during the last decade, there is good reason for optimism.
Furthermore, there are now several groups and international collaborations around the world applying CL methods to many systems, which
generate a wider range of situations than ever before from which insight can be gained. The most remarkable step forward, in the latest chapter
in the history of CL, is the derivation of conditions for correctness and how they relate to behavior at the boundaries of the integration region (at
infinity and at zeros of the complex weight). This new understanding has spawned new practical solutions such as gauge cooling, dynamic
stabilization, and modified actions (regulators), which have enabled more applications than previously thought possible.
Continued studies of CL will help illuminate when the method is reliable and when it is not. A more detailed understanding of the structure of
the problems in CL might help develop new methods to ameliorate or solve those problems.

%\subsection{Remaining Questions and Concerns}
%The application of CL to systems which suffer from a sign problem remains an area of active research. Concerns still remain about the
%consistency of the method, and the regimes in which it can be accurately used. The existence of anomalies which cause the method to
%converge to incorrect results is perhaps the question of greatest concern, as for the moment these anomalies cannot be predicted
%~\cite{PRDSalcedoCL2016}.

%One such study has recently suggested a mechanism for convergence to incorrect solutions, as well as a way to monitor that
%mechanism~\cite{PhysRevD.99.014512}. This work re-examines a model system previously studied in \cite{BERGES2008306},
%and by Stamatescu in unpublished notes: the system with complex action $S = i \beta \cos(x)$. The result of the CL
%simulation on this model is a well-converged set of results which are incorrect (but are equivalent to the stationary solution of the
%FP equation). The problem arises from the boundary terms of the Langevin time-dependent expectation values, which
%do not vanish for arbitrarily large Langevin time. While the boundary terms are not themselves accessible to general lattice
%calculations, the initial value of that boundary term appears to set the maximum of the boundary values, and so can help establish
%correctness of the results.

%\subsection{Comparison with other methods}
%Other methods exist to circumvent the sign problem, which were discussed in some detail in section~\ref{genformal}, all of which
%break down in particular regions. Complex Langevin can be checked against some of these methods in regions where both methods
%are applicable, and can serve as a benchmark for others in regimes where accuracy of the other method is unclear. This has already
%been examined in a comparison of CL with reweighting in lattice QCD. The two methods break down in different regions, with CL failing
%for small values of the coupling ($\beta$) and multi-parameter reweighting breaking down when the ratio of chemical potential and
%temperature fall into a specific range, $\mu/T \approx 1 - 1.5$~\cite{PRD92094516}. Additionally, phase-quenched approaches involve
%calculations which are in principle simple, and they work in some cases to compare against CL in situations where the sign problem is
%mild~\cite{Lattice2012Aarts}. More examples exists of areas where alternative methods for circumventing the sign problem can be used
%to study the method and confirm or refine its results.

%\subsection{The Future of Complex Langevin}


\end{document}
